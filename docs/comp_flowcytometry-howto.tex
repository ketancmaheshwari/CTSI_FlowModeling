%---------------------------------------------------
%	PACKAGES AND OTHER DOCUMENT CONFIGURATIONS
%---------------------------------------------------

\documentclass[landscape,a0paper,fontscale=0.285]{baposter} % Adjust the font scale/size here
\title{Cheat Sheet}
\usepackage[brazilian]{babel}
\usepackage[utf8]{inputenc}

\usepackage{graphicx} % Required for including images
\graphicspath{{figures/}} % Directory in which figures are stored

\usepackage{xcolor}
\usepackage{colortbl}
\usepackage{tabu}
%\usepackage{hyperref}

\usepackage{mathtools}
%\usepackage{amsmath} % For typesetting math
\usepackage{amssymb} % Adds new symbols to be used in math mode

\usepackage{booktabs} % Top and bottom rules for tables
\usepackage{enumitem} % Used to reduce itemize/enumerate spacing
\usepackage{palatino} % Use the Palatino font
\usepackage[font=small,labelfont=bf]{caption} % Required for specifying captions to tables and figures

\usepackage{multicol} % Required for multiple columns
\setlength{\columnsep}{1.5em} % Slightly increase the space between columns
\setlength{\columnseprule}{0mm} % No horizontal rule between columns

\usepackage{tikz} % Required for flow chart
\usetikzlibrary{decorations.pathmorphing}
\usetikzlibrary{shapes,arrows} % Tikz libraries required for the flow chart in the template
\setlist[itemize]{leftmargin=*}
\setlist[enumerate]{leftmargin=*}
\newcommand{\compresslist}{ % Define a command to reduce spacing within itemize/enumerate environments, this is used right after \begin{itemize} or \begin{enumerate}
\setlength{\itemsep}{1pt}
\setlength{\parskip}{0pt}
\setlength{\parsep}{0pt}
}

\definecolor{lightblue}{rgb}{0.145,0.6666,1} % Defines the color used for content box headers
\definecolor{pittblue}{RGB}{20,41,87} % Defines the color used for content box headers
\definecolor{pittgold}{RGB}{205,184,125} % Defines the color used for content box headers

\newcommand*{\R}{\texttt{ R }}

\begin{document}

\begin{poster}
{
 headerborder=closed, % Adds a border around the header of content boxes
 colspacing=0.8em, % Column spacing
 bgColorOne=white, % Background color for the gradient on the left side of the poster
 bgColorTwo=white, % Background color for the gradient on the right side of the poster
 borderColor=pittblue, % Border color
 headerColorOne=pittblue, % Background color for the header in the content boxes (left side)
 headerColorTwo=pittgold, % Background color for the header in the content boxes (right side)
 headerFontColor=white, % Text color for the header text in the content boxes
 boxColorOne=white, % Background color of the content boxes
 textborder=roundedleft, % Format of the border around content boxes, can be: none, bars, coils, triangles, rectangle, rounded, roundedsmall, roundedright or faded
 eyecatcher=true, % Set to false for ignoring the left logo in the title and move the title left
 headerheight=0.1\textheight, % Height of the header
 headershape=roundedright, % Specify the rounded corner in the content box headers, can be: rectangle, small-rounded, roundedright, roundedleft or rounded
 headerfont=\Large\bf\textsc, % Large, bold and sans serif font in the headers of content boxes
 %textfont={\setlength{\parindent}{1.5em}}, % Uncomment for paragraph indentation
 linewidth=2pt % Width of the border lines around content boxes
}
%----------------------------------------------------------------
%	TITLE
%----------------------------------------------------------------
{\bf\textsc{Computational Flow Cytometry "How To"}\vspace{0.5em}} % Poster title
{\textsc{C o m p u t a t i o n a l \ \ \ \ \ F l o w C y t o m e t r y \ \ \ \ \ " H o w - T o " \hspace{12pt}}}
{\textsc{Dept. of Immunology and Center for Research Computing \\ University of Pittsburgh \hspace{12pt}}} 

%------------------------------------------------
% ACCESS
%------------------------------------------------
\headerbox{Get Started}{name=access,column=0,row=0}{
This document describes how to do Computational Flow Cytometry.

We will process Flow Cytometry data to generate plots and diagrams.  We will
use the HTC cluster facility from the Center for Research Computing (CRC) at
Pitt.  We will use the Flow Cytometry data generated at the Department of
Immunology at Pitt. We will use some of the programs in \R language. The
programs are originally developed by the Saeys lab (see credits \& ack).

}

\headerbox{Logging in and Basics}{name=commoncommands, column=1,span=2, row=0}{
All the following commands need to be typed in a \textbf{terminal}. If you are
on a Mac OS, invoke the "terminal" program. You will see a \$ sign and a curser
waiting for you to type in commands. To execute, press enter after a command is
typed.

\colorbox[HTML]{cdb87d}{\makebox[\textwidth-2\fboxsep][l]{\bf - Connect}}
\texttt{ssh -l <yourpittid> htc.sam.pitt.edu \#connect to the HTC cluster }\\
Note that everything after the \# character is a comment and \textbf{is not} required to
be typed in when you try out the commands.\\
If you are connected to the HTC cluster, you will see a command prompt similar to:\\
\texttt{[login0a.htc.sam.pitt.edu ~]\$}

\colorbox[HTML]{cdb87d}{\makebox[\textwidth-2\fboxsep][l]{\bf - Data}}
Data may be moved from the cluster to your local computer as follows:\\
\texttt{scp afile.tgz <yourpittid>@htc.sam.pitt.edu:\textasciitilde/  \#copy from local computer to smp cluster }\\
\texttt{scp <yourpittid>@htc.sam.pitt.edu:\textasciitilde/afile.tgz .  \#copy from htc to local computer} \\
%------
}

%----------------------------------------------------------------
%	Help
%----------------------------------------------------------------
\headerbox{Get Help}{name=help,column=3,span=1,row=0}{
Does the FAQ answer your question?\\ \texttt{http://core.sam.pitt.edu/faqs}\\
Search the website:\\ \texttt{http://core.sam.pitt.edu/search}\\
Read the documentation\\
\texttt{http://core.sam.pitt.edu/node/6}\\
%Seek help \\ 
%\linebreak
%% {lp{5.8cm}lp{1.0cm}|}
%-----
\colorbox[HTML]{cdb87d}{\makebox[\textwidth-2\fboxsep][l]{\bf - When submitting support ticket}}
\begin{enumerate}\compresslist
\item Provide a descriptive, specific title
\item Specify the cluster the ticket applies to
\item Provide directory location if applicable
\end{enumerate}
\texttt{{\small core.sam.pitt.edu/node/add/support-ticket}}
\linebreak\\
%------
\colorbox[HTML]{cdb87d}{\makebox[\textwidth-2\fboxsep][l]{\bf - Contact / Feedback}}
Please send your feedback and suggestions for improvement to this document at \texttt{http://core.sam.pitt.edu/contact}
%-----------------------------------
}

%----------------------------------------------------------------
%	Software
%----------------------------------------------------------------
\headerbox{Common Commands}{name=software,column=0,row=1, below=access}{
\texttt{\textbf{ls} \#list files and dirs}\\
\texttt{\textbf{pwd} \#print current dir path}\\
\texttt{\textbf{cd <adir>} \#change to dir <dir>}\\
\texttt{\textbf{cd } \#change to the home dir}\\
\texttt{\textbf{cat <afile>} \#display contents of <afile>}\\
\texttt{\textbf{cp file1 file2} \#copy file1 to file2}\\
\texttt{\textbf{mv file1 file2} \#rename file1 to file2}\\
\texttt{\textbf{mv file1 <adir>/} \#move file1 to <dirname>}\\
\texttt{\textbf{touch afile} \#create a new file}\\
\texttt{\textbf{rm <filename>} \#delete a file <afilename>}\\
\texttt{\textbf{mkdir <adir>} \#create a new dir}\\
\texttt{\textbf{man rm} \#read manual of rm}\\
\texttt{\textbf{grep 'word' afile} \#search for 'word' in afile}\\
\texttt{\textbf{exit} \#exit out of the terminal}\\
\texttt{\textbf{nano afile} \#edit afile using nano}\\
}

%----------------------------------------------------------------
%	Hardware
%----------------------------------------------------------------
\headerbox{Quick Step-by-Step Guide}{name=hardware,column=1, span=2, row=1, below=commoncommands}{
\begin{enumerate}
\item Open a command prompt terminal on your computer, usually the terminal program on Mac OS
\item Log in to the \texttt{HTC} cluster at CRC: \texttt{ssh -l <pittID> htc.sam.pitt.edu}
\item Copy the directory containing \R programs, SLURM job scripts and borghesilab data (\textbf{one-time step}): \texttt{cp -r /ihome/sam/ketan/borghesilab \$HOME/}
\item Change into the \texttt{scripts} directory: \texttt{cd borghesilab/home-experiments/scripts}
\item Submit the first set of programs as SLURM job to the cluster: \texttt{sbatch set1.sbatch}
\item Periodically check of the job is finished: \texttt{squeue -u \$(whoami)}
\item Once the job is finished move the results back to your local computer
\item If the job resulted in errors, take a look at the troubleshooting section to see if problem may be fixed.
\item Wait for job to finish, ie. no job entry to be found as output of the \texttt{squeue -u \$(whoami)} command.
\item The results of the computation are saved in the \texttt{\textbf{results}} directory.
\item To copy the results back to your computer from the HTC cluster, open a new terminal on your computer and run the following command:\\
\texttt{scp -r <pittID>@htc.sam.pitt.edu:~/borghesilab/home-experiments/results .}
\end{enumerate}
%Set 1 contains the following programs:
%\begin{enumerate}\compresslist
%\item plotsettings
%\item preprocessing
%\item QC
%\end{enumerate}
%Set 2 contains the following programs:
%\begin{enumerate}\compresslist
%\item SPADE
%\item Rtsne
%\item FlowSOM
%\end{enumerate}
%Set 3 contains the following programs:
%\begin{enumerate}\compresslist
%\item FlowMeans
%\item FlowPeaks
%\item FlowClust
%\item Immunoclust
%\item samSPECTRAL
%\item FlowSOM Meta
%\end{enumerate}
%Set 4 contains the following programs:
%\begin{enumerate}\compresslist
%\item citrus
%\end{enumerate}
}
\headerbox{Credits \& Acknowledgements}{name=credits,column=3, below=help}{
We acknowledge the Saeys lab for the original \R programs for the data
	processing. We have made changes to the \R programs to adapt to the current
	data from Department of Immunology.  }


\end{poster}
\newpage

%%%%%%%%%%%%%%%%%%%%%%%%%%%%%%%%%%%%%%%%%%%%%%%%%%%%%%%%%%
%%%%%%%%%%%%%%%%%%    SECOND PAGE    %%%%%%%%%%%%%%%%%%
%%%%%%%%%%%%%%%%%%%%%%%%%%%%%%%%%%%%%%%%%%%%%%%%%%%%%%%%%%

\begin{poster}
{
borderColor=pittblue, headerColorOne=pittblue, headerColorTwo=pittgold,
	headerborder=closed, colspacing=0.8em, bgColorOne=white, bgColorTwo=white,
	headerFontColor=white, boxColorOne=white, textborder=roundedleft,
	eyecatcher=true, headerheight=0.1\textheight, headershape=roundedright,
	headerfont=\Large\bf\textsc, linewidth=2pt
}

%----------------------------------------------------------------
%	TITLE SECTION 
%----------------------------------------------------------------
%{\bf\textsc{Computational Flow Cytometry "How To"}\vspace{0.5em}} % Poster title
{\textsc{C o m p u t a t i o n a l \ \ \ \ \ F l o w C y t o m e t r y \ \ \ \ \ " H o w - T o " \hspace{12pt}}}
{\textsc{Dept. of Immunology and Center for Research Computing \\ University of Pittsburgh \hspace{12pt}}} 

\headerbox{Computation}{name=computation,column=0}{
CRC clusters use the SLURM scheduler. \\
\colorbox[HTML]{cdb87d}{\makebox[\textwidth-2\fboxsep][l]{\bf - Job Management}}
\texttt{sinfo}\hfill \#view info about nodes\\
\texttt{sbatch set1.sbatch}\hfill  \#submit a job \\
\texttt{squeue -u \$(whoami)}\hfill \#check job status \\
\texttt{scancel 12345}\hfill \#cancel job with id 12345\\
\texttt{squeue -j 12345 -o \%S}\hfill \#expected or actual start time of job id 12345\\
\colorbox[HTML]{cdb87d}{\makebox[\textwidth-2\fboxsep][l]{\bf - Job Status Codes}}
\texttt{PD} \dotfill  Pending \\
\texttt{R}  \dotfill  Running \\
\texttt{CA} \dotfill  Cancelled \\
\texttt{F} \dotfill   Failed \\
\texttt{CF} \dotfill   Configuring \\
\texttt{TO} \dotfill   Timed Out \\
\texttt{PR} \dotfill   Preempted \\
\texttt{NF} \dotfill   Node Failed \\
\texttt{S} \dotfill   Suspended \\
\texttt{CG} \dotfill  Completing \\
\texttt{CD} \dotfill  Completed \\
}

\headerbox{Notes}{name=notes, below=computation, column=0,row=1}{
\begin{itemize}\compresslist
\item To print this document on letter size paper use the printer's \textbf{fit to size} option.
\item The source code for all the programs and a \LaTeX\: file of this document may be found on git: \texttt{github.com/\\ketancmaheshwari/borghesilab}.
\end{itemize}
\colorbox[HTML]{cdb87d}{\makebox[\textwidth-2\fboxsep][l]{\bf - Acronyms}}
\texttt{HTC} \dotfill  High Throughput Computing\\
\texttt{CRC} \dotfill  Center for Research Computing\\
}

%------------------------------------------------
% Example SLURM Script
%------------------------------------------------

\headerbox{SLURM Scripts}{name=exampleslurm,column=1,span=2,row=0}{
\colorbox[HTML]{cdb87d}{\makebox[\textwidth-2\fboxsep][l]{\bf - set1}}
\#!/bin/bash\\
\#SBATCH -N 1 \\
\#SBATCH -{}-job-name=set1\\
\#SBATCH -{}-output=set1.out\\ 
\#SBATCH -{}-error=set1.err \\
\#SBATCH -{}-time=00:20:00 \\
\#SBATCH -{}-cpus-per-task=8 \\
\#SBATCH -{}-mem=32g \\
module purge\\
module load R\\
module load gsl\\ 
Rscript 1\_plotSettings.R\\
echo "Finished 1\_plotSettings.R"\\
Rscript 1\_preprocessing.R\\
echo "Finished 1\_preprocessing.R"\\
Rscript 1\_QC.R\\
echo "Finished 1\_QC.R"\\
echo "Set 1 finished"\\

\colorbox[HTML]{cdb87d}{\makebox[\textwidth-2\fboxsep][l]{\bf - set2}}
\#!/bin/bash\\
\#SBATCH -N 1 \\
\#SBATCH --job-name=set2\\
\#SBATCH --output=set2.out\\ 
\#SBATCH --error=set2.err \\
\#SBATCH --time=00:40:00 \\
\#SBATCH --cpus-per-task=8 \\
\#SBATCH --mem=32g \\
module purge\\
module load R\\
module load gsl\\ 
Rscript 2\_Rtsne.R\\
echo "Finished 2\_Rtsne.R"\\
Rscript 2\_SPADE.R\\
echo "Finished 2\_SPADE.R"\\
Rscript 2\_SPADE\_10000.R\\
echo "Finished 2\_SPADE\_10000.R"\\
echo "Set 2 finished"
}

\headerbox{Troubleshoot}{name=troubleshoot,column=3,row=0}{
\colorbox[HTML]{cdb87d}{\makebox[\textwidth-2\fboxsep][l]{\bf - ssh connection}}
\begin{itemize}\compresslist
	\item If you are on a wireless network, make sure the \textbf{VPN} connection is established.
	\item Use \texttt{ping} to check network connectivity to host, eg. \texttt{ping htc.sam.pitt.edu}
	\item Use ssh in verbose mode with \texttt{-v} to identify possible causes, eg. \texttt{ssh -v h2p.crc.pitt.edu}
\end{itemize}
\colorbox[HTML]{cdb87d}{\makebox[\textwidth-2\fboxsep][l]{\bf - jobs submission}}
\begin{itemize}\compresslist
	\item Sanity test the environment by submitting a simple job, eg.\\
\texttt{
\#!/bin/bash\\
\#SBATCH --output=test.out\\ 
\#SBATCH --error=test.err\\ 
\#SBATCH -t 00:10:00\\
srun echo "Hello \$(hostname)"
}
	\item Check output of \texttt{squeue -t PD} and \texttt{smap} 
\end{itemize}
%\colorbox[HTML]{cdb87d}{\makebox[\textwidth-2\fboxsep][l]{\bf - data access}}

\colorbox[HTML]{cdb87d}{\makebox[\textwidth-2\fboxsep][l]{\bf - Linux commands}}
\begin{itemize}\compresslist
\item Copying and pasting the error message into Google often returns helpful results on common pitfalls.
\item One common cause of Linux commands failing is that they do not find files
in the path where they are expecting them. Make sure to verify the files are in
the location where they are specified in the programs or as command line
arguments.
\end{itemize}
}
%----------------------------------------------------------------
%	REFERENCES  {name=objectives,column=0,row=0}
%----------------------------------------------------------------
%\headerbox{bb}{name=references,column=1,row=0}{}
%----------------------------------------------------------------
%	FUTURE RESEARCH
%----------------------------------------------------------------
%\headerbox{aa}{name=futureresearch,column=1,row=0}{}
%----------------------------------------------------------------
%	CONTACT INFORMATION
%----------------------------------------------------------------
%\headerbox{Contact Information}{name=contact,column=2,span=2,row=0}{}
%----------------------------------------------------------------
\end{poster}
\end{document}

